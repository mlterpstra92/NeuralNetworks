%!TEX root = report.tex
\subsection{Time variance}
One of the dangers of virtually every learning algorithm is overfitting. 
In this case, the trained model is too specific for the data and cannot correctly predict previously unforeseen observations because the generalization error increases as a result to a better fit to the training data. 
When the model is overfitted, it is highly unbiased towards the given data because it views all data as `typical'.
As a result, the variance beyond the dataset is very large; the small fluctuations in the dataset are given too much meaning, causing it to incorrectly label data outside the trainingset.

One way to prevent overfitting is to perform the action of early stopping. 
This is a form of regularization to prevent overfitting. 
When early stopping is applied the number of iterations of algorithms as gradient descent is limited when it is detected that the machine is overfitting. 

Unfortunately, there are no formal methods to determine when to stop, as there are only ad-hoc manners to prevent overfitting. 
In our cases, we observe that overfitting can occur as soon as after 500 iterations (see Figure~\ref{fig:overfitting}). 
Therefore, one way to perform early stopping for our dataset is to monitor the generalization error after 500 iterations.
If it increases as opposed to the previous iteration, the algorithm could decide to stop further training as it is starting to overfit.

\subsection{Example size variance}
Except for time variance as described in the section above, another interesting research case is varying the number of training examples. 
This has been shown in Figures~\ref{fig:costs_P100}-\ref{fig:costs_P4000}.
These figures show different situations: There is too little training data, enough training data, and even more training data. 

In the case of Figure~\ref{fig:costs_P100} one could argue that the training data is not representative for the entire dataset. 
This causes that the odds of overfitting is higher than with more training data because outliers have a larger influence on the training than with a larger training set.

In the case of Figure~\ref{fig:costs_P200} one could argue that in this case there is reasonably enough data to use the network.
We state that in this case that there is a fair balance between the risk of overfitting, the significance to the verification of the network, and the execution time of the training.

In the case of Figure~\ref{fig:costs_P500} we argue that the training dataset is even better.
A too large training set implies a too small test set.
In our case, the test set is still 90 \% of the entire dataset, so the test set is still a good representation of the dataset.
The effect of a too small test set is that it bears too little statistical significance for the verification of the classification of the neural network.

Figure~\ref{fig:costs_P4000} shows a nice convergence of the training error and test error.
We see that the errors converge almost immediately, compared to the other training sets.

Another consequence of using a too large training set is that the execution time of the training is very large, so one has to be careful with choosing \(P\).