%!TEX root = report.tex
We have implemented the Rosenblatt perceptron algorithm in MATLAB. We started the simulation with 25 neurons with $\alpha = 0.75, 1, 1.25 \dots 3$. Initially we generate $n_d = 50$ different patterns for our perceptron to classify. The maximimum number of iterations before we determine it is not linearly separable (whether it is or not) is $n_{max} = 100$. This means we generate per number of patterns, $N_s$, $n_d$ random patterns and try per pattern at most $n_{max}$ attempts to determine it is linearly separable. 

The patterns are, as mentioned before, generated randomly. The datasets are generated in one step as a $P \times N$ matrix of random values, generated by \texttt{randn}. At the end of each row a random label is appended which is generated using \verb randi . We map the output value, which is either 1 or 2, to either -1 or 1.

For each dataset we run the Rosenblatt algorithm as described in the practical assignment. For each pattern it is determined if the weights can classify the data according to its label. If it cannot, the weights are updated to come closer to the solution. If the weights do not change for all the patterns we determine that we determine that the perceptron is trained. Otherwise we keep on separating until $n_{max}$ is reached. Then we determine that the dataset is not linearly separable. 