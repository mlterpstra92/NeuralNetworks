%!TEX root = report.tex
The perceptron is one of the most basic building blocks for neural networks.
A perceptron has \(N\) input neurons, and just a single output neuron (called the label) that can have values \(\pm1\).
All input neurons are connected to the output neuron, using adaptive weights.
The input neurons are written as \(\xi\).
The label (output neuron) is written as \(S\).

We're regarding multiple patterns bundled in a dataset:
\[ \mathbb{D} = \left\{ \xi^\mu, S^\mu \right\} _{\mu=1}^{P} \]

where \(P\) is the number of patterns in the dataset, and \(\mu\) is the index variable that addresses the different patterns.

The input neurons \(\xi^\mu\) are modelled as vectors in \(\mathbb{R}^N\), and are generated as random Gaussian components between zero and one.

The labels \(S^\mu\) are generated independently to be \(\pm1\) with equal probability.

Rosenblatt's perceptron learning algorithm\cite{rosenblatt1958perceptron} is used to train perceptrons.
In essence, it works by adapting the neuron weights to keep the perceptron's data consistent with the pattern data.

The neuron weight \(w\) of the perceptron are initialized to zero.